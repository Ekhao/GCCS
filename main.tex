\documentclass{article}

\usepackage{todonotes}

\title{Generic Clock Cycle Simulator}
\author{Emil Njor \& Peter Jensen}
\date{\today}


\begin{document}
\maketitle
In the up and coming field of embedded machine learning, it is hard to manually design machine learning models that are both accurate and efficient.
One possible solution to this problem is to make a computer search for such machine learning models.
A computer will be much faster at exploring different machine learning models than a human engineer and can be used by people that are not experts in machine learning.
This is known in the research community as hardware-aware neural architecture search (HW-NAS) and is a subfield of AutoML.

Most works in HW-NAS search for machine learning models that are accurate and have a low amount of floating-point operations (FLOPs).
Using the metric of floating-point operations as a proxy for efficiency has however been heavily criticized in the community.
The main criticism is that the number of floating-point operations will often not be a good proxy for the efficiency metrics that are important in embedded systems such as inference time, memory usage, and energy consumption.
Inference time is a measure of how long it takes to run a trained machine learning model on a given hardware platform.

In this project, we aim to develop a simulator that can be used to estimate the amount of clock cycles executing a machine learning model on given hardware platform will take.
The output of such a simulator, i.e., a clock cycle metric, is a better proxy for inference time, and energy consumption than the number of floating-point operations.
E.g., an estimate of the inference time of a machine learning model on a given hardware platform can be calculated by multiplying the number of clock cycles with the clock speed of the hardware platform.
The energy consumption can also be estimated based on the size of the hardware platform and the power consumption per clock cycle.

We expect that the developed simulator will work both for scalar processors and different digital signal processors (DSP) for machine learning.
The simulator should be parametric such that it can support typical DSP configurations, such as VLIW (Very Long Instruction Word) and SIMD (Single Instruction/Multiple Data) architectures.
Depending on the quality of the project, we may be interested in using the simulator in an academic publication and potentially publishing the simulator as an open-source project to help other researchers.




\end{document}
